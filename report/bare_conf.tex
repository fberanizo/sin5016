\documentclass[conference]{IEEEtran}

\usepackage{cite}
\usepackage{amsmath}
\usepackage{algorithmic}
\usepackage{url}
\usepackage[utf8]{inputenc}

\hyphenation{}

\begin{document}
\title{Relatório do Trabalho 1\\ de Aprendizado de Máquina}

\author{\IEEEauthorblockN{Fábio Beranizo Fontes Lopes}
\IEEEauthorblockA{Escola de Artes, Ciências e Humanidades (EACH)\\
Universidade de São Paulo (USP)\\
Email: f.lopes@usp.br}}

\maketitle


Este relatório tem como objetivo mostrar os procedimentos e resultados do Trabalho 1 da disciplina de Aprendizado de Máquina. O propósito dessa tarefa foi testar diferentes algoritmos em atividades de reconhecimento biométrico. Duas técnicas de aprendizado supervisionado foram escolhidas para comparação: Multilayer Perceptron (MLP) e Support Vector Machines (SVM).
% Ambos algoritmos foram implementados em Python.

As seções a seguir explicam as atividades de pré-processamento do conjunto de dados, o treinamento dos modelos e uma discussão dos resultados obtidos.

\section{Base de Dados}
A base de dados utilizada neste trabalho foi a Homologous Multi-modal Traits Database da Shandong University (SDUMLA-HMT) \cite{sdumlahtm}. Este conjunto de dados contém faces de 106 indivíduos em 7 diferentes ângulos. Estas imagens foram capturadas com diferentes poses, expressões faciais, acessórios e iluminações, assemelhando-se a uma aplicação real. 

Cada amostra deste conjunto é rotulada com o indivíduo que aparece na foto, tornando possível aplicar técnicas de aprendizado supervisionado para classificação.

\subsection{Pré-processamento}
Originalmente, as imagens do SDUMLA-HTML possuem resolução 480x640 e estão no formato RGB. Antes de realizar a extração de características das imagens, a região de interesse foi segmentada aplicando-se o Algoritmo Viola-Jones. Em uma segunda etapa, as imagens foram convertidas para escala de cinza a sua dimensão alterada para 128x128.

\subsection{Transformada Wavelet}
A extração de caracterísicas das imagens utilizou a transformada Wavelet. Esta técnica de processamento de sinal realiza a análise de um dado sinal no domínio do tempo e frequência \cite{costa2011ensemble}. Uma função Wavelet pode ser descrita como uma função que apresenta média zero e obedece a Equação:

\[\int_{-\infty}^{\infty} \Psi(t)dt = 0\]

Para utilizar a transformada Wavelet em imagens 2D as linhas e colunas são tratadas como sinais independentes. Ao final da decomposição Wavelet 2D são geradas 4 subimagens. Três delas (HL, LH, HH) contém os detalhes horizontais, verticais e diagonais da imagem original. Uma outra (LL) é uma aproximação da imagem original.

Ao final deste processo são gerados coeficientes que podem ser utilizados para representar a imagem original. Os coeficentes da imagem LL são geralmente os mais utilizados, pois apresentam menor dimensão e boa representatividade \cite{burrus1997introduction}.

\section{Classificadores}
Não é necessário explicar a teoria referente aos algoritmos de aprendizado de máquina
escolhidos quando esses foram apresentados em aula. Algoritmos de aprendizado de máquina
extra, estratégias de combinação e de melhoraria dos algoritmos devem ser explicadas em
detalhes no relatório.

Duas técnicas de aprendizado supervisionado foram implementadas para este trabalho: MLP e SVM. Estas técnicas não serão decritas em detalhe, pois já foram apresentadas em aula.

A comparação dos algoritmos utilizou 5x2 Fold Cross-validation

 A avaliação de ambos os algoritmos foi feita por meio do
F-score médio, visto que os dados são desbalanceados, obtido
por cada experimento, tal que o melhor experimento foi

25\% do conjunto de dados para teste


Dois algoritmos de classificação foram implementados
para o trabalho: o K-NN e a MLP. As próximas
subseções descrevem, em linhas gerais, as implementações,
procedimentos realizados para os testes e os resultados
obtidos por cada algoritmo de mineração de dados.
A avaliação de ambos os algoritmos foi feita por meio do
F-score médio, visto que os dados são desbalanceados, obtido
por cada experimento, tal que o melhor experimento foi
aquele que obteve o melhor F-score médio.
Para calcular o F-score, foi gerada uma matriz de confusão
geral e, com ela, foram geradas matrizes especı́ficas para
cada classe observada. Desse modo, para cada classe, foram
calculadas as quantidades de verdadeiros positivos, falsos
positivos, falsos negativos e verdadeiros negativos e, com
estes dados, calculou-se a acurácia, erro, especificidade,
revocação, precisão e F-score, este dado pela fórmula:

\subsection{SVM}
Máquinas de Vetores suporte.
Explicar que foi utilizado One-vs-all


\subsection{MLP}
Multilayer perceptron.
Explicar método de definição de alfa ótimo

\subsection{Ensembles}
Ver a tese do aluno do clodoaldo.

\section{Resultados}

\section{Conclusão}


%\begin{figure}[!t]
%\centering
%\includegraphics[width=2.5in]{myfigure}
%\caption{Simulation results for the network.}
%\label{fig_sim}
%\end{figure}

%\begin{figure*}[!t]
%\centering
%\subfloat[Case I]{\includegraphics[width=2.5in]{box}%
%\label{fig_first_case}}
%\hfil
%\subfloat[Case II]{\includegraphics[width=2.5in]{box}%
%\label{fig_second_case}}
%\caption{Simulation results for the network.}
%\label{fig_sim}
%\end{figure*}

%\begin{table}[!t]
%% increase table row spacing, adjust to taste
%\renewcommand{\arraystretch}{1.3}
%\caption{An Example of a Table}
%\label{table_example}
%\centering
%\begin{tabular}{|c||c|}
%\hline
%One & Two\\
%\hline
%Three & Four\\
%\hline
%\end{tabular}
%\end{table}

\bibliography{bibliography}

\end{document}