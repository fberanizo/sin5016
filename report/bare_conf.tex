\documentclass[conference]{IEEEtran}

\usepackage{cite}
\usepackage{amsmath}
\usepackage{algorithmic}
\usepackage{url}
\usepackage[utf8]{inputenc}

\hyphenation{}

\begin{document}
\title{Relatório do Trabalho 1\\ de Aprendizado de Máquina}

\author{\IEEEauthorblockN{Fábio Beranizo Fontes Lopes}
\IEEEauthorblockA{Escola de Artes, Ciências e Humanidades (EACH)\\
Universidade de São Paulo (USP)\\
Email: f.lopes@usp.br}}

\maketitle


Este relatório tem como objetivo mostrar os resultados do trabalho 1 desenvolvido para a disciplina de Aprendizado de Máquina.


\section{Conjunto de Dados}

%\begin{figure}[!t]
%\centering
%\includegraphics[width=2.5in]{myfigure}
%\caption{Simulation results for the network.}
%\label{fig_sim}
%\end{figure}

%\begin{figure*}[!t]
%\centering
%\subfloat[Case I]{\includegraphics[width=2.5in]{box}%
%\label{fig_first_case}}
%\hfil
%\subfloat[Case II]{\includegraphics[width=2.5in]{box}%
%\label{fig_second_case}}
%\caption{Simulation results for the network.}
%\label{fig_sim}
%\end{figure*}

%\begin{table}[!t]
%% increase table row spacing, adjust to taste
%\renewcommand{\arraystretch}{1.3}
%\caption{An Example of a Table}
%\label{table_example}
%\centering
%\begin{tabular}{|c||c|}
%\hline
%One & Two\\
%\hline
%Three & Four\\
%\hline
%\end{tabular}
%\end{table}

\bibliography{bare_conf}

\end{document}